\documentclass[11pt]{report}

\usepackage{graphicx}
\usepackage[utf8]{inputenc}


\begin{document}
\title{Virtual Reality}
\author{Maximilian Sieß}
\maketitle

\tableofcontents

%\chapter{Abstract}

\chapter{Introduction}
Virtual Reality is the attempt to use technology, such as head mounted display devices, and computer generated graphics, to allow the user to experience a sense of presence in a virtual environment. This is used in a wide variety of cases, including but not limited to, entertainment, education, medical therapy, research, and visualization.

\chapter{Related Work}
	\section{Headmounted Display Technology}
		The concept of Virtual Reality dates back to the 1980s (Citation Needed), and unlike some depiction of it in pop culture at the time, never archived the level of immersion and presence that recent technological advances enable us to. Due to a jump in interest hardware such as the Oculus Rift found funding in recent years. In the case of Oculus, it was via Crowdfunding. But as a result, commercial products from Google, Valve/HTC or Sony have been announced. At the time of writing, none of the mentioned companies have released a commercially available end user product. Oculus Rift has released and sold Developer Kits, which are most frequently used in modern Virtual Reality endeavours.
	\includegraphics[scale=0.3]{or_small.png}

	\section{Software}
		Programming software for virtual reality does not differ much from regular computer graphics programming. Most commercial vendors offer their own API that helps translating a virtual camera to a two camera 3D setup. It was found however, that how the camera is used is imperitive to not give the user of the virtual reality headsead motion sickness. For example, moving the camera without the user moving their head was resulted in severly negative feedback from the test subjects.
	
	\section{Additional Hardware}
	With headmounted displays, vision, the groundwork for a feeling of presence in virtual reality, is laid out. Headsets or surround sound systems have been shown to suffice for the audio representation of the virtual environment.\\
	Moving around naturally has proven difficult, however. While video game demos often use a gamepad, it is less than ideal for upholding a sense of presence. Products, such as the XYZ try to enable free movement in virtual reality.



	%\section{Entertainment}
	%Oculus rift, Valve, Project Morpheus - VIDJA GAMES!
	
	%\section{Education}
	%Look at papers you downloaded
	
	%\section{Therapy}
	%Look at papers or download more!
	
	%\section{Research}
	%Papers!
	
	%\section{Visualization}
	%You know, like for surgery, architecture and stuffs.


\chapter{Discussion}
%How helpful is it now? Will it be real reality soon? how soon? absolute presence is not possible with just a headmount set, so wtf Valve, where is my absolute-immersion-set that lets me LittlePip and shoot down some raiders, eh?


%\chapter{Conclusion}

\chapter{References}

%\chapter{Appendix}

\bibliographystyle{plain}
\bibliography{vr_bib}

\chapter{notes}
Overview of VR - Definition of Virtual Reality, Immersion, Perception and Telepresence. 
"Virtual Reality development were manly found in the military and academic research until technologies became more cost-effective."
Oh yeah, also input devices are a big factor, shit. Gloves, wands (like the Wii or PS Move) and computer vision.
Also, military wants VR too, woops.
	
Best Practices - VR is hard and every single minor detail is super fucking important so listen to us, fuckers!
"Acceleration creates a mismatch among your visual, vestibular, and proprioceptive senses;
minimize the duration and frequency of such conflicts. Make accelerations as short (preferably
instantaneous) and infrequent as you can."

arm therapy - Use VR to distract patience from pain while their burn wounds are treated. Seems to work!

unwanted sideeffects - cybersickness or simulator sickness

supply chain eduction - use a VR game to teach

serious games for ancient manuscripts - making vr games about really old books ? Allowing people to experience... reading a really old book. \cite{zyda05}

\end{document}
